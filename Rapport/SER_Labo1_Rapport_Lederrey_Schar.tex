\documentclass[a4paper]{article}
\usepackage[utf8]{inputenc}
\usepackage[T1]{fontenc}
\usepackage[french]{babel}
\usepackage{listings}
\usepackage{color}
\usepackage[usenames,dvipsnames,svgnames,table]{xcolor}
\usepackage{titlesec}
\usepackage{tikz}
\usetikzlibrary{trees}
\usetikzlibrary{decorations.pathmorphing}
\usetikzlibrary{decorations.markings}


% Title Page
\title{SER - Labo1 \\ Sérialisation textuelle avec XML \& JSON}
\author{Yann Lederrey \& Joel Schär}


    \lstset{
    	tabsize=3,
    	inputencoding=utf8,
    	frame=lines,
    	caption=Test,
    	label=code:sample,
    	frame=shadowbox,
    	rulesepcolor=\color{gray},
    	xleftmargin=20pt,
    	framexleftmargin=15pt,
    	keywordstyle=\color{blue},
    	commentstyle=\color{OliveGreen},
    	stringstyle=\color{red},
    	numbers=left,
    	numberstyle=\scriptsize,
    	numbersep=5pt,
    	breaklines=true,
    	showstringspaces=false,
    	basicstyle=\footnotesize,
    	emph={cinema, projections, films, acteurs, motsCle, genres, langages, critiques },emphstyle={\color{magenta}}
    }
    
    \colorlet{punct}{red!60!black}
    \definecolor{delim}{RGB}{20,105,176}
    \colorlet{numb}{magenta!60!black}
    
    \lstdefinelanguage{json}{
    	stepnumber=1,
    	numbersep=5pt,
    	showstringspaces=false,
    	breaklines=true,
    	literate=
    	*{0}{{{\color{numb}0}}}{1}
    	{1}{{{\color{numb}1}}}{1}
    	{2}{{{\color{numb}2}}}{1}
    	{3}{{{\color{numb}3}}}{1}
    	{4}{{{\color{numb}4}}}{1}
    	{5}{{{\color{numb}5}}}{1}
    	{6}{{{\color{numb}6}}}{1}
    	{7}{{{\color{numb}7}}}{1}
    	{8}{{{\color{numb}8}}}{1}
    	{9}{{{\color{numb}9}}}{1}
    	{:}{{{\color{punct}{:}}}}{1}
    	{,}{{{\color{punct}{,}}}}{1}
    	{\{}{{{\color{delim}{\{}}}}{1}
    	{\}}{{{\color{delim}{\}}}}}{1}
    	{[}{{{\color{delim}{[}}}}{1}
    	{]}{{{\color{delim}{]}}}}{1},
    }
    
    
    \definecolor{amber}{rgb}{1.0, 0.49, 0.0}
    \definecolor{jaune}{rgb}{1.0, 0.75, 0.0}
    \definecolor{amethyst}{rgb}{0.6, 0.4, 0.8}
    \definecolor{antiquebrass}{rgb}{0.8, 0.58, 0.46}
    \definecolor{aqua}{rgb}{0.0, 1.0, 1.0}
    \definecolor{babypink}{rgb}{0.96, 0.76, 0.76}
   
   %delete first line indent
   \setlength{\parindent}{0in}
    
\begin{document}
\maketitle
\pagebreak

\tableofcontents
\pagebreak

\section{Introduction}
Ce rapport détail les mises en forme que nous avons choisie pour notre structure des documents XML et JSON. Nous illustrons notamment cette structure sous forme d'arbre avec un choix de couleurs différenciant chaque type d'élément.

\section{XML}
\subsection{Graph XML}
\tikzset{
	photon/.style={decorate, decoration={snake}, draw=green},
	normal/.style={draw=black}
}
\subsubsection{Explication graphiques}
Chaque Noeud cinema correspond au noeud parent du fichier XML.\\
Les noeuds bleu correspondent aux éléments.\\
Les noeuds verts correspondent à des attributs.\\
Les noeuds colorés correspondent à des Id (ID et FK liés selon la couleur).

\begin{center}
	\begin{tikzpicture}[
	attribut/.style={rectangle,draw,fill=green!20},
	normal/.style={rectangle,draw,fill=blue!20,rounded corners=.8ex},
	filmid/.style={rectangle,draw,fill=amber!50,rounded corners=.8ex},
	acteurid/.style={rectangle,draw,fill=jaune!50,rounded corners=.8ex},
	genreid/.style={rectangle,draw,fill=amethyst!50,rounded corners=.8ex},
	critiqueid/.style={rectangle,draw,fill=antiquebrass!50,rounded corners=.8ex},
	motcleid/.style={rectangle,draw,fill=aqua!50,rounded corners=.8ex},
	langageid/.style={rectangle,draw,fill=babypink!50,rounded corners=.8ex},
	root/.style={rectangle,draw,fill=red!20},
	transform canvas={scale=0.8},
	sibling distance=7em,]
	\node [root]{cinema}
	child {node[normal] {projections} [normal]
		child { node[normal] {projection} [normal]
			child { node[normal] {dateProjection} [normal]}
			child { node[normal] {numeroSalle} [normal]}
			child { node[normal] {filmProj} [normal]
				child { node[filmid] {filmId} [normal]}
			}
			child { node[normal] {acteursProj} [normal]
				child { node[acteurid] {acteurProj} [normal]
					child { node[attribut] {acteurProjId} [normal]}
					child { node[attribut] {nomRole} [normal]}
					child { node[attribut] {placeRole} [normal]}
				}
			}
		}
	};
	\end{tikzpicture}
\end{center}

\vspace{170px}

\begin{center}
	\begin{tikzpicture}[
	attribut/.style={rectangle,draw,fill=green!20},
	normal/.style={rectangle,draw,fill=blue!20,rounded corners=.8ex},
	filmid/.style={rectangle,draw,fill=amber!50,rounded corners=.8ex},
	acteurid/.style={rectangle,draw,fill=jaune!50,rounded corners=.8ex},
	genreid/.style={rectangle,draw,fill=amethyst!50,rounded corners=.8ex},
	critiqueid/.style={rectangle,draw,fill=antiquebrass!50,rounded corners=.8ex},
	motcleid/.style={rectangle,draw,fill=aqua!50,rounded corners=.8ex},
	langageid/.style={rectangle,draw,fill=babypink!50,rounded corners=.8ex},
	root/.style={rectangle,draw,fill=red!20},
	transform canvas={scale=0.8},
	sibling distance=7em,]
	\node [root]{cinema}
	child {node[normal] {films} [normal]
		child { node[normal] {film} [normal]
			child { node[filmid] {filmId} [normal]}
			child { node[normal] {titre} [normal]}
			child { node[normal] {synopsis} [normal]}
			child { node[normal] {duree} [normal]}
			child { node[normal] {critiqueFilm} [normal]
				child { node[critiqueid] {critiquesFilmId} [normal]}
			}
			child { node[normal] {genresFilm} [normal]
				child { node[normal] {genreFilm} [normal]
					child { node[genreid] {genreFilmId} [normal]}
				}
			}
			child { node[normal] {motsCleFilm} [normal]
				child { node[normal] {motCleFilm} [normal]
					child { node[motcleid] {motCleFilmId} [normal]}
				}
			}
			child { node[normal] {langagesFilm} [normal]
				child { node[normal] {langageFilm} [normal]
					child { node[langageid] {langageFilmId} [normal]}
				}
			}
			child { node[normal] {photo} [normal]}
		}
	};
	\end{tikzpicture}
\end{center}

\vspace{170px}

\begin{center}
	\begin{tikzpicture}[
	attribut/.style={rectangle,draw,fill=green!20},
	normal/.style={rectangle,draw,fill=blue!20,rounded corners=.8ex},
	filmid/.style={rectangle,draw,fill=amber!50,rounded corners=.8ex},
	acteurid/.style={rectangle,draw,fill=jaune!50,rounded corners=.8ex},
	genreid/.style={rectangle,draw,fill=amethyst!50,rounded corners=.8ex},
	critiqueid/.style={rectangle,draw,fill=antiquebrass!50,rounded corners=.8ex},
	motcleid/.style={rectangle,draw,fill=aqua!50,rounded corners=.8ex},
	langageid/.style={rectangle,draw,fill=babypink!50,rounded corners=.8ex},
	root/.style={rectangle,draw,fill=red!20},
	transform canvas={scale=0.8},
	sibling distance=7em,]
	\node [root]{cinema}
	child { node[normal] {acteurs} [normal]
		child { node[normal] {acteur} [normal]
			child { node[acteurid] {acteurId} [normal]}
			child { node[normal] {nom} [normal]}
			child { node[normal] {nomNaissance} [normal]}
			child { node[normal] {biographie} [normal]}
			child { node[normal] {sexe} [normal]
				child { node[attribut] {type(enum)} [normal]}
			}
			child { node[normal] {dateNaissance} [normal]}
			child { node[normal] {dateDeces} [normal]}
		}
	};
	\end{tikzpicture}
\end{center}

\vspace{170px}

\begin{center}
	\begin{tikzpicture}[
	attribut/.style={rectangle,draw,fill=green!20},
	normal/.style={rectangle,draw,fill=blue!20,rounded corners=.8ex},
	filmid/.style={rectangle,draw,fill=amber!50,rounded corners=.8ex},
	acteurid/.style={rectangle,draw,fill=jaune!50,rounded corners=.8ex},
	genreid/.style={rectangle,draw,fill=amethyst!50,rounded corners=.8ex},
	critiqueid/.style={rectangle,draw,fill=antiquebrass!50,rounded corners=.8ex},
	motcleid/.style={rectangle,draw,fill=aqua!50,rounded corners=.8ex},
	langageid/.style={rectangle,draw,fill=babypink!50,rounded corners=.8ex},
	root/.style={rectangle,draw,fill=red!20},
	transform canvas={scale=0.8},
	level 1/.style={sibling distance=6cm},
	level 2/.style={sibling distance=3cm}, 
	level 3/.style={sibling distance=2cm},]
	\node [root]{cinema}
	child { node[normal] {motsCleFilm} [normal]
		child { node[normal] {motCle} [normal]
			child { node[motcleid] {motCleId} [normal]}
			child { node[normal] {labelMc} [normal]}
		}
	}
	child { node[normal] {genres} [normal]
		child { node[normal] {genre} [normal]
			child { node[genreid] {genreId} [normal]}
			child { node[normal] {labelGe} [normal]}
		}
	}
	child { node[normal] {langages} [normal]
		child { node[normal] {langage} [normal]
			child { node[langageid] {langageId} [normal]}
			child { node[normal] {labelLa} [normal]}
		}
	}
	child { node[normal] {critiques} [normal]
		child { node[normal] {critique} [normal]
			child { node[critiqueid] {critiqueId} [normal]}
			child { node[normal] {texte} [normal]}
			child { node[normal] {note} [normal]}
		}
	};
	\end{tikzpicture}
\end{center}

\vspace{170px}

\subsection{Fichier DTD}
\lstinputlisting[language=XML, caption={pathe.dtd}]{../pathe.dtd}
\subsection{Exemple XML}
\lstinputlisting[language=XML, caption={pathe.xml}]{../pathe.xml}
\subsection{Validation XML}
\lstinputlisting[language=json, caption={validation}]{../validation.txt}

\section{JSON}
\subsection{Graph JSON}
\subsection{Exemple JSON}
\lstinputlisting[language=json, caption={pathe.json}]{../pathe.json}

\section{conclusion}
Ce travail nous a permis de mieux comprendre la structure des formats XML et JSON. Nous avons expérimenter l'utilité d'avoir une grammaire DTD qui évite les inconsistances dans le document XML et permet de valider l'existence de tous les éléments. Nous n'avons pas rencontré de grand difficulté quand à la réalisation de ce travail.


\end{document}          
	